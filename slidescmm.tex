\documentclass{beamer}






\usetheme{uic}
\usepackage{amsfonts,amsmath,oldgerm,algorithm,algpseudocode}
\usepackage{graphicx,multicol}
\usepackage{jigsaw}
\usepackage[font=small,labelfont=bf]{caption} % Required for specifying captions to tables and figures
\usepackage{array}
\usepackage{qrcode}
\usepackage{listings}




\newcolumntype{C}[1]{>{\centering\arraybackslash}p{#1}@{}}

\newcommand{\hrefcol}[2]{\textcolor{uihteal}{\href{#1}{#2}}}
\newcommand{\testcolor}[1]{\colorbox{#1}{\textcolor{#1}{test}}~\texttt{#1}}

% Please see Section 18.1 of Beamer User Guide for all the options \usefonttheme provides
\usefonttheme[onlymath]{serif}
% \usefonttheme{serif} % use this if you would like Serif font throughout (and not just for math)


%\titlebackground*{assets/uic_lockup_blue.pdf}
\titlebackground*{title.pdf}

% NOTE 1: The asterisk splits the background image. This option is good
% for logo-based backgrounds. If you use an image based background
% it's recommended to not split it:

% \titlebackground{assets/uic_seo.jpg}

% NOTE 2: If you use a title background that does not have a logo, you might
% want to enable logo in the top left.
% To do that, simply comment out this line
\themecolor{light}

\title{A Modular Development and Education Platform for Coordinate Measurement Solutions}
\subtitle{Using \LaTeX\ to prepare slides}
% This can be adjusted accordingly for longer titles
\setlength{\titleboxwidth}{0.50\textwidth}

\authortitle{M.Sc.}
\author{\href{mailto:stefan.helmert@htw-dresden.de}{Stefan Helmert}}
\institute{Faculty of Informatics / Mathematics}

\date{\today}

\begin{document}
	\maketitle
	\themecolor{light} % reverts to a logo based theme (if you disabled it for title page)
	
	% enabled after title page creation (i.e. after \maketitle)
	\footlinecolor{white}
	
	
	
\begin{frame}[fragile]{Coordinate Measurement Machine}
	\framesubtitle{What is it? In which areas is it used? How is it operated?}
	\includegraphics[height=5cm]{imgs/Patent_US8438746B2}
	\includegraphics[height=5cm]{imgs/Patent_US20050151963A1}
	\begin{multicols}{3}
	\begin{itemize}
		\item Tool and Die Verification
		\item Surface and Profile Measurement
		\item Reverse Engineering
		\item Quality Assurance \& Statistical Process Control
		\item Calibration of Gauges and Fixtures
		\item Prototyping and R\&D
	\end{itemize}
	\end{multicols}
\end{frame}
	
	
\begin{frame}[fragile]{OSH CMM}
	\framesubtitle{PoC first check}
	\includegraphics[height=8cm]{imgs/OSHCMMold}
\end{frame}

\begin{frame}[fragile]{OSH CMM}
	\framesubtitle{Evoloved version}
	\includegraphics[height=8cm]{imgs/OSHCMMcomplete}
\end{frame}
	
\begin{frame}[fragile]{Open Source Hardware}
	\framesubtitle{OSH as a necessity not as a nice-to-have}
	Fill market gap only achievable with OSH 

	\begin{itemize}
		\item educate
		\item customize
		\item integrate
	\end{itemize}
\end{frame}
	
\begin{frame}[fragile]{Architecture}
	\framesubtitle{The modular setup of the data path}
	\includegraphics[height=8cm]{imgs/architecture}
\end{frame}
	
\begin{frame}[fragile]{Design for Openness}
	\framesubtitle{Tear the barriers down!}
	
	
	\begin{columns}
	\begin{column}{0.5\textwidth}
		
	\begin{itemize}
		\item complete
		\item diagnosable
		\item adaptable
		\item tolerant
		\item simple
		\item understandable
	\end{itemize}

		
	\end{column}
	\begin{column}{0.5\textwidth}  %%<--- here
		
	\begin{itemize}
		\item reproducible
		\item reusable
		\item transparent
		\item safe
		\item interoperable
		\item accessible
	\end{itemize}
	
		
	\end{column}
\end{columns}
	
	\vspace{12mm}

	\textbf{Openness as the core value}

	We need OSH designed with an open source mindset, not just publishing proprietary concepts.
\end{frame}
	
\begin{frame}[fragile]{Complete}
	\framesubtitle{The whole ecosystem is available.}
	
\begin{tikzpicture}[scale=3]
	
\path (2,-2) pic[
fill=yellow, draw=black, thick,
scale=3, pic text={Software},
pic text options={text=black}
]{piece={1}{1}{0}{0}};
	
	
\path (2,-3) pic[
fill=green, draw=black, thick,
scale=3, pic text={Mechanics},
pic text options={text=black}
]{piece={0}{-1}{-1}{0}};	


\path (3,-2) pic[
fill=cyan, draw=black, thick,
scale=3, pic text={Firmware},
pic text options={text=black}
]{piece={1}{0}{0}{-1}};	

\path (3,-3) pic[
fill=pink, draw=black, thick,
scale=3, pic text={Electronics},
pic text options={text=black}
]{piece={0}{0}{-1}{1}};	
	
	
\node[circle, draw=black, fill=lightgray, minimum size=1.5cm] at (3,-2) {Tools};
\end{tikzpicture}
	
\end{frame}
	
\begin{frame}[fragile]{Accessible}
	\framesubtitle{No proprietary requirements}
	\begin{columns}
		\begin{column}{0.5\textwidth}
			\begin{itemize}
				\item OSS tools
				\begin{itemize}
					\item \emph{Arduino IDE}
					\item \emph{KiCad}
					\item \emph{FreeCad}
					\item \emph{xyzCad}
					\item \emph{OrcaSlicer}
				\end{itemize}
				\item Cheap hardware tools
				\begin{itemize}
					\item 3d printer
					\item Soldering iron
					\item Screwdriver
				\end{itemize}
			\end{itemize}
		\end{column}
		\begin{column}{0.5\textwidth}  %%<--- here
			\begin{itemize}
				\item Manufacturing service friendly
				\item Low craft skill requirements
				\begin{itemize}
					\item Mounting
					\item No machining
					\item No welding
					\item No drilling
					\item No lasercutting
					\item Soldering $\rightarrow$ PCBA service
					\item 3d printing $\rightarrow$ 3d printing service
				\end{itemize}
			\end{itemize}
		\end{column}
	\end{columns}

\end{frame}

\begin{frame}[fragile]{Simple}
	\framesubtitle{Plug and Play, default operation, stand alone mode}
	\includegraphics[width=.95\linewidth]{imgs/host_software_1.png}
\end{frame}


\begin{frame}[fragile]{Interoperable}
	\framesubtitle{Adds value to third party infrastructure.}
	
	\begin{columns}
		\begin{column}{0.3\textwidth}
			
	\begin{itemize}
		\item available software
		\item Replace rotary encoders
		\item Retrofit for old CMMs
		\item \emph{Arduino} with network
	\end{itemize}
			
			
		\end{column}
		\begin{column}{0.7\textwidth}  %%<--- here
			
		\includegraphics[width=.95\linewidth]{imgs/logicboard}
			
		\end{column}
	\end{columns}
	


\end{frame}

\begin{frame}[fragile]{Safe}
	\framesubtitle{No risky requirements during the whole life cycle}
	\begin{tabular}{@{} *{2}{C{.5\linewidth}} }
		\includegraphics[width=.9\linewidth]{imgs/encoderfront} & \includegraphics[width=.9\linewidth]{imgs/stepup} \\[\abovecaptionskip]
		No machining & Ready-made step-up converters
	\end{tabular}
\end{frame}

\begin{frame}[fragile]{Safe -- Reusable -- Tolerant}
	\framesubtitle{Low cost, low risk}
	\begin{tabular}{@{} *{2}{C{.5\linewidth}} }
		\includegraphics[width=.9\linewidth]{imgs/usb} & \includegraphics[width=.8\linewidth]{imgs/SpiroBrixx} \\[\abovecaptionskip]
		5V USB only \emph{Arduino} & 3d printed \emph{SpiroBrixx} parts
	\end{tabular}
\end{frame}

\begin{frame}[fragile]{Tolerant}
	\framesubtitle{Robust against mistreatment}
	\begin{tabular}{@{} C{0.75\linewidth} C{0.3\linewidth} @{} }
		\includegraphics[width=.95\linewidth]{imgs/protectionschem} & \includegraphics[width=.95\linewidth]{imgs/protection} \\[\abovecaptionskip]
		Electrical protection of the rotary encoder & PCB top side
	\end{tabular}
\end{frame}
	
\begin{frame}[fragile]{Diagnosable}
	\framesubtitle{Self-test circuits are part of the main logic board.}
	\includegraphics[width=.95\linewidth]{imgs/diagnoseschem}
\end{frame}
	
\begin{frame}[fragile]{Adaptable}
	\framesubtitle{Foundation for further development}

	\textbf{Mechanics} parameterized implicit surface geometry
	\begin{itemize}
		\item implicit chamfers
		\item AI-ready
		\item compact
		\item free-style shapes
	\end{itemize}


	\begin{lstlisting}[language=Python]
@njit
def fz_thread_profile(pnt, a=1.0):
	y = pnt[1]
	if abs(y) > 2 * a:
	return y
	x = pnt[0] * 2 * np.pi
	sg = np.sign(np.sin(x))
	snglr = 2**-20 - 2**-19 * sg
	p = a / (-np.sin(x) / 2 + snglr)
	q = y / (np.sin(x) + snglr) - 1.0
	return a * (-p - sg * np.sqrt(max(0, p**2 - 4 * q)))
	\end{lstlisting}
	


\end{frame}
	
\begin{frame}[fragile]{Build and Contribute}
	\framesubtitle{Follow me on GitHub}
	
	
	\begin{columns}
		\begin{column}{0.6\textwidth}

		\qrcode[height=5cm]{https://github.com/TheTesla/HTWD-OSH-CMM-Arm}

\vspace{5mm}

\url{https://github.com/TheTesla/HTWD-OSH-CMM-Arm} 


		\end{column}
		\begin{column}{0.4\textwidth}  %%<--- here

		M.Sc. \\
\textbf{Stefan Helmert} \\
Faculty of Informatics / Mathematics \\
T +49 351 462 2166 \\
M +49 177 850 6921 \\
stefan.helmert@htw-dresden.de \\

\vspace{10mm}

\textbf{Funded by} \\

\vspace{5mm}

\includegraphics{imgs/logo-vwstiftung-dark} \\

\vspace*{8mm}


		\end{column}
	\end{columns}
	
	
	

	
	
	
\end{frame}
	
	
	
								

\end{document}
